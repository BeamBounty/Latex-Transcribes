\documentclass{article}

\usepackage{amsmath} 

\begin{document}
    
\title{Math 214}
\author{Section 1.1}
\date{}
\maketitle % Needed to make Title and Author work properly

\section*{1A}  
 
    \begin{align}
        x_1 - 3x_2 & = 2  \\ % Equation 1 for 1a
        2x_2  & = 6  % Equation 2 for 1a
    \end{align}

\raggedright From Equation 2 we get, \newline

    \begin{math}
         2x_2 = 6 
         \newline % For some reason double \\ doesn't work
         x_2 = 6/2 
         \newline
         x_2 = 3
    \end{math} \newline

\raggedright We can now plug in that value, \newline

    \begin{math}
         x_1 - 3(3) = 2  
         \newline 
         x_1 - 9 = 2
         \newline
         x_1 = 11
    \end{math} \newline

\raggedright Thus the final answer (s) are, \newline
    \begin{math}
        x_1 = 11 \newline
        x_2 = 3 
    \end{math}

% BELOW IS FOR QUESTION 1B
% BELOW IS FOR QUESTION 1B
% BELOW IS FOR QUESTION 1B

\section*{1B}   

    \setcounter{equation}{0} % Used to reset equation counter
    
    \begin{align}
        x_1 + x_2 + x_3 & = 8 \\ % Equation 1 for 1b
        2x_2 + x_3 & = 5 \\ % Equation 2 for 1b
        3x_3 & = 9 \\ % Equation 3 for 1b
    \end{align}

\raggedright From Equation 3 we get, \newline

    \begin{math}
         3x_3 = 9 
         \newline % For some reason double \\ doesn't work
         x_3 = 9/3 
         \newline
         x_3 = 3
    \end{math} \newline

\raggedright We can now plug in  \begin{math} x_3 = 3 \end{math} into Equation 2 \newline

    \begin{math}
         2x_2 + 3 = 5  
         \newline 
         2x_2 = 2
         \newline
         x_2 = 1
    \end{math} \newline

\raggedright Plug in \begin{math} x_3 = 3 \end{math} and \begin{math} x_2 = 1 \end{math}  into Equation 1, \newline
    
    \begin{math}
        x_1 + 1 + 3 = 8 \newline
        x_1 + 4  = 8 \newline
        x_1 = 4  
    \end{math}

\raggedright Thus your final answer is, 

    \begin{math}
        x_1 = 4 \newline
        x_2 = 1 \newline
      x_3 = 3 
    \end{math}

    % SECTION 1C
    % SECTION 1C
    % SECTION 1C

\section*{1C}   

\setcounter{equation}{0} % Used to reset equation counter
    
    \begin{align} 
        x_1 + 2x_2 + 2x_3 + x_4 & = 5 \\ % Equation 1 for 1C
        3x_2 + x_3 - 2x_4 & = 4 \\ % Equation 2 for 1C
        -x_3 + 2x_4 & = -1 \\ % Equation 3 for 1C
        4x_4 & = 4 \\ % Equation 4 for 1C 
    \end{align}

\raggedright From Equation 4 we get, \newline

    \begin{math}
         4x_4 = 4 
         \newline % For some reason double \\ doesn't work
         x_4 = 1 
         \newline
    \end{math} \newline

\raggedright We can now plug in  \begin{math} x_4 = 1 \end{math} into Equation 3 \newline

    \begin{math}
         -x_3 + 2(1) = -1  
         \newline 
         x_3 = 2 + 1
         \newline
         x_3 = 3
    \end{math} \newline

\raggedright Plug in \begin{math} x_4 = 1 \end{math} and \begin{math} x_3 = 3 \end{math}  into Equation 2, \newline
    
    \begin{math}
        3x_2 + 3 - 2(1) = 1 \newline
        3x_2 + 1  = 1 \newline
        3x_2 = 0 \newline
        x_2 = 0 \newline  
    \end{math}

\raggedright Finally plug in \begin{math} x_4 = 1 \end{math} , \begin{math} x_3 = 3 \end{math} and \begin{math} x_2 = 0 \end{math} into Equation 1, \newline

    \begin{math}
        x_1 + 2(0) + 2(3) + 1 = 5 \newline
        x_1 + 0 + 6 + 1 = 5 \newline
        x_1 + 7 = 5 \newline
        x_1 = -2 \newline
    \end{math}

    \raggedright Thus your final results are, \newline

    \begin{math}
        x_4 = 1 \newline
        x_3 = 3 \newline
        x_2 = 0 \newline
        x_1 = -2 
    \end{math}

% START OF SECTION 1D
% START OF SECTION 1D
% START OF SECTION 1D

\section*{1D}   

    \setcounter{equation}{0} % Used to reset equation counter
    
    \begin{align} 
        x_1 + x_2 + x_3 + x_4 + x_5 & = 5 \\ % Equation 1 for 1D
        2x_2 + x_3 - 2x_4 + x_5 & = 1 \\ % Equation 2 for 1D
        4x_3 + x_4 - 2x_5 & = 1 \\ % Equation 3 for 1D
        x_4 - 3x_5 & = 0 \\ % Equation 4 for 1D
        2x_5 & = 2 \\ % Equation 5 for 1D
    \end{align}

\raggedright From Equation 5 we get, \newline

    \begin{math}
         2x_5 = 2 
         \newline % For some reason double \\ doesn't work
         x_5 = 1 
         \newline
    \end{math} \newline

\raggedright Plug in \begin{math} x_5 = 1 \end{math} into Equation 4 \newline

    \begin{math}
         x_4 - 3(1) = 0  
         \newline 
         x_4 - 3 = 0
         \newline
         x_4 = 3
    \end{math} \newline

\raggedright Now, plug in \begin{math} x_4 = 3 \end{math} and \begin{math} x_5 = 1 \end{math}  into Equation 3, \newline
    
    \begin{math}
        4x_3 + 3 - 2(1) = 1 \newline
        4x_3 + 1  = 1 \newline
        4x_3 = 0 \newline
        x_3 = 0 \newline  
    \end{math}

\raggedright With these, plug in \begin{math} x_3 = 0 \end{math} , \begin{math} x_4 = 3 \end{math} and \begin{math} x_5 = 1 \end{math} into Equation 2, \newline

    \begin{math}
        2x_2 + 0 - 2(3) + 1 = 1 \newline
        2x_2 - 5 = 1 \newline
        2x_2 = 4 \newline
        x_2 = 2 \newline
    \end{math}

\raggedright Finally, plug in \begin{math} x_2 = 2 \end{math} , \begin{math} x_3 = 0 \end{math} , \begin{math} x_4 = 3 \end{math} and \begin{math} x_5 = 1 \end{math} into Equation 1, \newline
    
    \begin{math}
        x_1 + 2 + 0 + 3 + 1 = 5 \newline
        x_1 + 6 = 5 \newline
        x_1 = -1 \newline 
    \end{math}

\raggedright Thus, your final answers are: \newline

    \begin{math}
        x_1 = -1 \newline
        x_2 = 2 \newline
        x_3 = 0 \newline
        x_4 = 3 \newline
        x_5 = 1 \newline
    \end{math}

% START OF MATRIXES
% START OF MATRIXES
% START OF MATRIXES

\begin{center}{\section*{\huge{Coefficient Matrixes}}}\end{center}

\section*{2A}

\setcounter{equation}{0}

\begin{align}
    \hspace*{1.0in} 
    x_1 - 3x_2 &= 2& \\
    2x_2 &= 6&
\end{align}

\raggedright Thus the coefficient matrix would be,
\[
\begin{pmatrix}
1 && -3 \\
0 && 2 \\    
\end{pmatrix}
\]

% SECTION 2B
% SECTION 2B
% SECTION 2B

\section*{2B}

\setcounter{equation}{0}

\begin{align}
    \hspace*{1.0in}
    x_1 + x_2 + x_3 &= 8& \\
    2x_2 + x_3 &= 5& \\
    3x_3 &= 9&
\end{align}

\raggedright Thus the coefficient matrix would be,

\[
\begin{pmatrix}
    1 && 1 && 1 \\
    0 && 2 && 1 \\
    0 && 0 && 3
\end{pmatrix}
\]



\end{document}